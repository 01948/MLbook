\section{Mixture of Experts (Смесь экспертов)}

Модель \textit{Смеси экспертов} (Квазилинейный ансамбль, Mixture of Experts, MoE) является архитектурой машинного обучения, которая сочетает в себе несколько моделей (экспертов) для решения сложных задач. Основная идея заключается в том, чтобы разделить пространство входных данных на части, в каждом из которых определенный эксперт специализируется. Общая модель обучается так, чтобы комбинировать выходы экспертов с учетом их специализации.

Математически, выход модели MoE для признакового описания объекта $\mathbf{x}$ может быть представлен как:

$$
y = \sum_{k=1}^{K} g_k(\mathbf{x}) f_k(\mathbf{x}),
$$

где:
\begin{itemize}
    \item $K$ — количество экспертов,
    \item $f_k(\mathbf{x})$ — локальная модель, функция $k$-го эксперта, производящая прогноз,
    \item $g_k(\mathbf{x})$ — шлюзовая функция или функция компетентности, определяющая вес вклада $k$-го эксперта, причём $\sum_{k=1}^{K} g_k(\mathbf{x}) = 1$ и $g_k(\mathbf{x}) \geq 0$ для всех $k$.
\end{itemize}

Шлюзовая (Gate) функция обычно моделируется с помощью функции softmax

$$
g_k(\mathbf{x}) = \frac{\exp(h_k(\mathbf{x}))}{\sum_{j=1}^{K} \exp(h_j(\mathbf{x}))},
$$

где $h_k(\mathbf{x})$ — функция компетентности для $k$-го эксперта. Выбираются из каких-либо содержательных соображений.

Преимущество MoE заключается в способности моделировать сложные зависимости путем разделения задачи между специализированными экспертами, что может улучшить обобщающую способность и эффективность обучения.

\subsection{Задачи и решения}

\subsubsection{Задача 1}

Пусть имеется модель MoE с двумя экспертами, функции которых заданы как $f_1(x) = 2x$ и $f_2(x) = x^2$. Функции $h_k (x):$ $h_1 (x) = -x$, $h_2 (x) = x$.

Требуется найти выражение для общего выхода модели $y$ в зависимости от $x$.

\textbf{Решение:}

Шлюзовая функция моделируется как:

$$
g_1(x) = \frac{\exp(-x)}{\exp(-x) + \exp(x)}, \quad g_2(x) = \frac{\exp(x)}{\exp(-x) + \exp(x)}.
$$

Суммарный выход модели:

$$
y = g_1(x) f_1(x) + g_2(x) f_2(x) = g_1(x) \cdot 2x + g_2(x) \cdot x^2.
$$

Подставим выражения для $g_1(x)$ и $g_2(x)$:

$$
y = \frac{\exp(-x)}{\exp(-x) + \exp(x)} \cdot 2x + \frac{\exp(x)}{\exp(-x) + \exp(x)} \cdot x^2.
$$

Поэтому итоговое выражение для $y$:

$$
y = \frac{\exp(-x) \cdot 2x + \exp(x) \cdot x^2}{\exp(-x) + \exp(x)}.
$$

\subsubsection{Задача 2}

Рассмотрим модель MoE, где шлюзовая функция выбирает только одного эксперта в зависимости от знака $x$:
$$
g_1(x) = 
\begin{cases} 
1, & \text{если } x \geq 0, \\ 
0, & \text{если } x < 0,
\end{cases}
$$
$$
g_2(x) = 
\begin{cases} 
0, & \text{если } x \geq 0, \\ 
1, & \text{если } x < 0.
\end{cases}
$$
Функции экспертов заданы как $f_1(x) = x^2$ и $f_2(x) = -x$. Найдите общий выход модели $y$ для любого $x$.

\textbf{Решение:}

Поскольку в каждый момент времени активен только один эксперт, общий выход модели определяется функцией активного эксперта.

Для $x \geq 0$: 
$$
g_1(x) = 1, \quad g_2(x) = 0, \quad y = g_1(x) f_1(x) = x^2.
$$

Для $x < 0$: 
$$
g_1(x) = 0, \quad g_2(x) = 1, \quad y = g_2(x) f_2(x) = -x.
$$

Таким образом, общий выход модели равен:
$$
y = 
\begin{cases} 
x^2, & \text{если } x \geq 0, \\ 
-x, & \text{если } x < 0.
\end{cases}
$$
Это означает, что модель ведет себя по-разному на положительных и отрицательных значениях $x$, отражая специализацию экспертов на разных областях входных данных.

\subsubsection{Задача 3}

Рассмотрим модель Mixture of Experts, состоящую из двух экспертных моделей $f_1$ и $f_2$, а также шлюзовой функции $g$. Пусть на вход подаётся одно признаковое значение $x$. Выражения для выходов моделей заданы следующим образом:

$$
f_1(x) = w_1 x + b_1, \quad f_2(x) = w_2 x + b_2, \quad g(x) = \sigma(v x + c)
$$

где $\sigma(z) = \frac{1}{1 + e^{-z}}$ — сигмоидальная функция активации, а $w_1, w_2, b_1, b_2, v, c$ — параметры модели.

Выход всей модели MoE определяется как:

$$
y(x) = g(x) f_1(x) + (1 - g(x)) f_2(x)
$$

\textbf{Вопрос:} Предположим, что при $x = 0$ мы наблюдаем, что $y(0) = 1$, $f_1(0) = 1$, $f_2(0) = 3$. Найдите значение $g(0)$.

\textbf{Решение:}

Из условия задачи при $x = 0$ имеем:

$$
y(0) = g(0) \cdot f_1(0) + (1 - g(0)) \cdot f_2(0) = g(0) \cdot 1 + (1 - g(0)) \cdot 3
$$

Подставляем известное значение $y(0) = 1$:

$$
g(0) \cdot 1 + (1 - g(0)) \cdot 3 = 1
$$
$$
g(0) + 3 - 3g(0) = 1
$$
$$
-2g(0) + 3 = 1
$$
$$
-2g(0) = -2
$$
$$
g(0) = 1
$$

Таким образом, $g(0) = 1$.
