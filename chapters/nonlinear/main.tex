\section*{Методы Ньютона-Рафсона, Ньютона-Гаусса}

Мы уже познакомились с задачами линейной регрессии и обсудили несколько методов их решения. Но что делать если задача нелинейна? Оказывается, что идея локальной линейности гладкой функций позволяет свести задачу к более простому. На этой идее основаны методы второго порядка - Ньютона-Рафсона и Ньютона-Гаусса.

\subsection*{Метод Ньютона-Рафсона}

Начнем немного сдалека, а именно рассмотрим задачу поиска нуля $a$ функций $f(x)$. Пусть мы находимся в точке $a^{0}$ и хотим найти такое приращение $h$, чтобы приблизиться к точке $a$: $a^{0} + h \approx a$. Применим разложение в ряд:

\begin{gather*}
    f(a^{0} + h) = f(a^{0}) + f'(a^{0})h + o(h)\\[1em]
    f(a^{0} + h) \approx f(a) = 0 \Rightarrow f(a^{0}) + f'(a^{0})h \approx 0
\end{gather*}

Откуда

\[
h \approx - \frac{f(a^{0})}{f'(a^{0})}
\]

Значит, для поиска нуля выпуклой функций \( f \) можно применить следующий итеративный метод:

\[
a^{k + 1} = a^{k} - \frac{f(a^{k})}{f'(a^k)}
\]

Вернемся к начальной задаче:

\begin{itemize}
    \item обучающая выборка \( X^{l} = (x_{i}, y_{i}) \), где \( x_{i} \) - вектор признаков \( i \)-го объекта
    \item \( y_{i} = y(x_{i}),\quad y: X \to Y \) - неизвестная регрессионная зависимость
    \item \( f(x, \alpha) \) - нелинейная модель регрессии, где \( \alpha \) - вектор параметров
\end{itemize}

Хотим решить задачу оптимизации методом наименьших квадратов:

\[
Q(\alpha, X^{l}) = \sum\limits_{i = 1}^{l}\left(f(x_{i}, \alpha) - y_{i} \right)^{2} \to \min_{\alpha}
\]

Функция потерь \( Q(\alpha, X^{l}) \) выпукла и гладкая в предположений гладкости \( f(\alpha, x) \), поэтому для ее минимизаций достаточно найти нуль производной (градиента). Применяя рассуждения выше и опустив технические детали, получим следующий итерационный процесс:

\[
\alpha^{k + 1} = \alpha^{k} - h_{k}\left(Q''( \alpha^{k} )\right)^{-1} Q'(\alpha^{k})
\]

где \( Q'( \alpha^{k} ) \), \( Q''(\alpha^{k} ) \) - градиент и гессиан \( Q \) в точке \( \alpha^{k} \) соответственно, \( h_{k} \) - величина шага.
\subsection*{Метод Ньютона-Гаусса}

Подсчет обратного гессиана на каждой итерации может дорого обходиться, поэтому посмотрим на полученный выше результат с другой стороны. Для этого запишем несколько формул:

Компоненты градиента \( Q(\alpha, X^{l}) \):

\[
\frac{\partial Q(\alpha, X^{l})}{\partial\alpha_{j}} = 2 \sum\limits_{i = 1}^{l} \left(f(x_{i}, \alpha) - y_{i} \right)\frac{\partial f(x_{i}, \alpha)}{\partial\alpha_{j}}
\]

Компоненты гессиана:

\[
\frac{\partial Q(\alpha, X^{l})}{\partial\alpha_{j}\partial\alpha_{k}} = 2\sum\limits_{i = 1}^{l}\frac{\partial f(x_{i}, \alpha)}{\partial\alpha_{j}}\frac{\partial f(x_{i}, \alpha)}{\partial\alpha_{k}} - 2\sum\limits_{i = 1}^{l}\left(f(x_{i}, \alpha) - y_{i}\right)\frac{\partial^2 f(x_{i}, \alpha)}{\partial\alpha_{j}\partial\alpha_{k}}
\]

Второе слагаемое в формуле выше полагается равным нулю, исходя из линейной аппроксимации функций \( f \). Введем следующие обозначения и перепишем формулу итерации метода Ньютона-Рафсона:

\begin{gather*}
    F_{k} = \left( \frac{\partial f}{\partial\alpha_{j}}(x_{i}, \alpha^{k}) \right)_{i, j}\\[1em]
    f_{k} = \left( f(x_{i}, \alpha^{k}) \right)_{i}
\end{gather*}

Получим:

\[
\alpha^{k + 1} = \alpha^{k} - h_{k}(F_{k}^{T}F_{k})^{-1}F_{k}^{T}(f_{k} - y)
\]

Положив \( \theta = (F_{k}^{T}F_{k})^{-1}F_{k}^{T}(f_{k} - y) \), получим решение задачи линейной регрессии, где новые ответы обучающей выборки -- \( \left(f_{k} - y \right) \), с новой матрицей признаков~--~\( F_{k} \). Таким образом, каждый шаг метода Ньютона-Гаусса сводится к задаче линейной регрессии. 

\section*{Задачи}

\subsection*{Задача 1: Локальная сходимость}

Найдите допустимые значения начального приближения для поиска нуля функции \( f(x) = \frac{x}{\sqrt{1 + x^{2}}} \).

\textbf{Решение:}

Нуль функции \( f \) достигается в точке \( a = 0\). Посчитаем производную \( f \):

\[
f'(x) = \left(\frac{x}{\sqrt{1 + x^{2}}}\right)' = \frac{1}{\sqrt{1 + x^{2}}} - x\cdot\frac{2x}{2(1 + x^{2})^{3/2}} = \frac{1}{(1 + x^{2})^{3/2}}
\]

Распишем формулу итерации метода Ньютона:

\[
a^{k + 1} = a^{k} - \frac{f(a^{k})}{f'(a^{k})} = -(a^{k})^3
\]

Отсюда видно, что сходимость есть при \( |a^{0}| < 1 \). Делаем вывод о важности выбора начального приближения в данном методе.

\subsection*{Задача 2: Квадратичная задача}

Как сработает метод Ньютона-Рафсона для поиска минимума задачи \( f(x)~=~x^{T}Ax + bx + c\),  где $x, b \in \mathbb{R}^{n}$, $A$ - симметричная, положительно определенная матрица.

\textbf{Решение:}

Для поиска минимума нужно найти нуль градиента. Это и будет точкой минимума, так как задача выпукла.
Посчитаем градиент и гессиан:

\begin{gather*}
    \nabla f(x) = Ax + b\\[1em]
    \nabla^{2} f(x) = A
\end{gather*}

Пусть \( x^{0} \) начальная точка. Применяя формулу из метода Ньютона-Рафсона получим:

\[
x^{1} = x^{0} - \left(\nabla^{2} f(x^{0})\right)^{-1}\nabla f(x^{0}) = x^{0} - A^{-1}\left(Ax^{0} + b\right) = -A^{-1}b
\]

но с другой стороны, градиент обращается в нуль в этой точке:

\[
\nabla f(x) = 0 = Ax + b \Rightarrow x = -A^{-1}b
\]


Значит для квадратичной задачи данный метод дает ответ за 1 шаг. Вообще говоря, если функция $\mu$-выпукла и имеет $M$-липшецевый гессиан, то скорость сходимости локально квадратична.

\subsection*{Задача 3: Система уравнений}

Составьте алгоритм решения следующей системы с помощью методов второго порядка:

\[
\begin{cases}
x^{2} + y^{2} = 4\\
y = e^{x}
\end{cases}
\]

\textbf{Решение:}

Нужно найти нули следующей функции от двух переменных
\[
F(x, y) = 
\begin{pmatrix}
x^{2} + y^{2} - 4 \\
y - e^x
\end{pmatrix}
\]

Запишем итерацию метода Ньютона-Рафсона:

\[
x^{k + 1} = x^{k} - J_{F}(x^{k})^{-1}F(x^{k})
\]

Его можно записать как:

\[
J_{F}(x^{k})(x^{k + 1} - x^{k}) = -F(x^{k})
\]

Посчитаем матрицу якоби функции $F$:

\[
J_{F}(x) = 
\begin{pmatrix}
2x   & 2y \\
-e^x & 1
\end{pmatrix}
\]

Тогда на каждом шаге нужно решить следующую систему:
\[
\begin{pmatrix}
    2x^{k} & 2(y^{k})^{2} \\
    -e^{x^{k}} & 1 
\end{pmatrix}
\begin{pmatrix}
    c_{1}^{k + 1}  \\
    c_{2}^{k + 1}
\end{pmatrix} 
=
\begin{pmatrix}
    (x^{k})^2 + (y^{k})^2 - 4\\
    (y^{k}) - e^{x^{k}}
\end{pmatrix} 
\]

где \( c^{k + 1} = -(x^{k + 1} - x^{k})\). Тогда для решения данной задачи нужно взять начальную точку и проделать несколько итераций описанных уравнениями выше.

\section{Backfitting}

На практике встречаются задачи, в которых использование линейных моделей необосновано, но и не удаётся предложить явную нелинейную модель. В таком случае строится модель вида

$$\displaystyle y(x)=f(x,\theta)=\sum_{j=1}^m\theta_j \phi_j(x_j),$$

где $x$ ~--- объект, $x_j$ ~--- признаки объекта, $\phi_j:\mathbb{R}\rightarrow \mathbb{R}$ ~--- нелинейные в общем случае преобразования.

Задача состоит в том, чтобы одновременно подбирать коэффициенты модели $\theta_j$ и неизвестные преобразования $\phi_j$.

Суть метода backfitting (метод настройки с возвращением) заключается в чередовании оптимизации коэффициентов $\theta_j$ при постоянных $\phi_j$ методами линейной регрессии, и оптимизации преобразований $\phi_j$ при постоянных коэффициентах $\theta_j$. 

Для минимизации используется сумма квадратов ошибок на всех объектах:

$Q(\theta, \phi) = \displaystyle \sum_{i=1}^l \left( y(x_i) - y_i \right)^2 = \sum_{i=1}^l \left( \sum_{j=1}^m \theta_j \phi_j(X_{ij}) - y_i \right)^2 $

Здесь $X$ ~--- матрица признаков.

\subsection{Алгоритм backfitting}
\begin{algorithmic}
\State $\phi_j(t) \equiv t$ ~--- изначальное приближение линейными функциями
\While{($Q$ уменьшается)} 
    \State $\theta \gets \text{argmin}_{\theta}Q(\theta, \phi_j)$ ~--- при фиксированных $\phi_1,\dots, \phi_k$
    \For{$j=1\dots m$}
        \State $\displaystyle r_i = y_i - \sum_{k\neq j} \theta_k \phi_k(X_{ik})$
        \State $r_i$ ~--- ошибка модели на $i$ объекте, без учёта $j$ признака
        \State $\displaystyle\phi_j \gets \text{argmin}_{\phi} \sum_{i=1}^l \left( \theta_j \phi(X_{ij}) - r_i \right)^2$ ~--- при $\theta_j=const$
    \EndFor
\EndWhile 
\end{algorithmic}

Оптимизация по коэффициентам $\theta \gets \text{argmin}_{\theta}Q(\theta, \phi_j)$ выполняется методами линейной регрессии, такими как стохастический градиентный спуск.

Оптимизация по функции $\phi_j$ выполняется одномерными методами, такими как ядерное сглаживание.

Варианты дальнейшего развития метода:
\begin{itemize}
    \item [1.] Во внутреннем цикле выбирать индексы $j$ не в фиксированном порядке, а в первую очередь оптимизировать дающие наибольший вклад в $Q$.
    \item [2.] Регуляризация $Q$ по параметрам $\theta$ и по сложности функций $\phi$.
\end{itemize}

\subsection{Задача}

Если матрица признаков $X$ разреженная, с какой проблемой можно столкнуться, применяя алгоритм backfitting, и как её решить?

Решение: если каждая из функций $\phi_j$ вызывается лишь на небольшом наборе аргументов, то её значения на них могут быть почти независимы (если модель $\phi$ достаточно сложна, например, полином высокого порядка), и случится переобучение. Варианты решения: большая константа регуляризации по сложности функций $\phi$; ограничить $\phi$ более простым классом функций (например, только квадратичные функции вместо полиномов); расширить матрицу признаков несколькими простыми нелийными преобразованиями ($\sin x$, $x^2$...) и использовать методы линейной регрессии.

\section{Метод наименьших квадратов с итеративным пересчётом весов (IRLS)}

\textbf{Метод наименьших квадратов с итеративным пересчётом весов} (Iteratively Reweighted Least Squares) применяется для решения задач оптимизации. В частности, применение метода Ньютона-Рафсона к задаче \textit{Логистической регрессии} сводится к \textbf{IRLS}.

Напомним постановку задачи линейной регрессии. Будем искать приближенное решение системы:

\[ \begin{bmatrix} 
    a_{11} & a_{12} & \dots  & a_{1N} \\
    \vdots & \vdots & \ddots & \\
    a_{M1} & \dots  & \dots  & a_{MN}
\end{bmatrix} \begin{bmatrix} 
    x_{1} \\
    \vdots \\
    x_{M} \\ 
\end{bmatrix}
=
\begin{bmatrix} 
    y_{1} \\
    \vdots \\
    y_{M} \\
\end{bmatrix} \]

То же самое в матричных обозначениях:

\[
    \boldsymbol{A} \boldsymbol{x} = \boldsymbol{y}
\]

Поставим задачу минимизировать норму вектор ошибки (невязки):

\[
    \boldsymbol{e} = \boldsymbol{A} \boldsymbol{x} - \boldsymbol{y} 
\]

В качестве нормы возьмем p-норму: $||\boldsymbol{e}||_p = \left( \sum_i |e_i|^p \right)^{1/p}$. Как известно, если методом наименьших квадратов можно аналитически найти решение, минимизирующее Евклидову норму вектора невязки (root-mean-squared error) $\sqrt{\boldsymbol{e}^T \boldsymbol{e}}$. Покажем как можно построить итеративный подход, использующий результаты для взвешенного метода наименьших квадратов, для нахождения оптимального решения для $l_p$ нормы.

\subsection{Взвешенные метод наименьших квадратов}

Для построения IRLS нужно сперва вспомнить основные результаты взвешенной модификации МНК. Добавим в Евклидову норму веса для каждой компоненты вектора $\boldsymbol{x}$ и будем минимизировать норму:

\[
    \boldsymbol{||W e||_2^2} = \sum_i w_i^2 e_i^2 = \boldsymbol{e}^T \boldsymbol{W}^T \boldsymbol{W} \boldsymbol{e} 
\]

\[
    \boldsymbol{W} = diag(w_1, w_2, \dots, w_M)
\]

$\boldsymbol{W}$ - диагональная матрица ненулевых весов. Легко видеть, что такое взвешивание соответствует линейному преобразованию растяжения с коэффициентами $w_1, w_2, \dots w_M$. Для переопределенной системы решение с минимальной взвешенной нормой оказывается равным:

\[
    \boldsymbol{x} = \left[ \boldsymbol{A^T W^T W A} \right]^{-1} \boldsymbol{A^T W^T W y}
\]

\subsection{Алгоритм \textbf{IRLS}}

Поиск минимума $||\boldsymbol{e}||_p = \left( \sum_i |e_i|^p \right)^{1/p}$ сводится к итеративному алгоритму, где на каждом шаге применяется взвешенный МНК. Набор весов $w_1, w_2, \dots, w_M$ пересчитывается на каждой итерации $n$.

\[
||e(n +1)||_p = \left( \sum_i w_i^2(n) |e_i(n)|^2 \right)^{1/2}
\]

\[
w_i(n) = e_i(n)^{\frac{p - 2}{2}}
\]

Начальные веса $w_1, w_2, \dots, w_M$ берутся единичными, т.е. для первой итерации используется в качестве приближения используется стандартный МНК.

\subsection{Задачи}

\subsubsection{Задача 1: WLS}

Получить оптимальное решение для переопределенной системы с взвешенной нормой. \\

\textbf{Решение:}

Вспомним результат обычного МНК:

\[
    \boldsymbol{A} \boldsymbol{x} = \boldsymbol{y}
\]

\[
    \boldsymbol{x} = \left[\boldsymbol{X^T X}\right]^{-1} \boldsymbol{X y}
\]

Домножая слева на $\boldsymbol{W}$ получим: $\boldsymbol{W} \boldsymbol{A} \boldsymbol{x} = \boldsymbol{W} \boldsymbol{y}$. Получаем исходную задачу МНК с матрицей $\boldsymbol{W} \boldsymbol{A}$ и правой частью $\boldsymbol{W} \boldsymbol{y}$. Отсюда сразу получается ответ:

\[
    \boldsymbol{x} = \left[ \boldsymbol{A^T W^T W A} \right]^{-1} \boldsymbol{A^T W^T W y}
\]

\subsubsection{Задача 2: Пример IRLS для $l_1$}

Реализовать IRLS и применить его для нахождения линейной модели по набору точек $x, y$: $(1, 2), (2, 3), (4, 6)$. Применить реализованный метод и убедиться, чтобы итеративный метод сходится к $3 / 2$ для $p = 1$ (норма $l_1$) и к $32 / 21$ для $p = 2$.

\subsubsection{Задача 3: Зависимость весов для $l_p$}

Обосновать выбор $w_i(n) = e_i(n)^{\frac{p - 2}{2}}$, считая что итеративный метод сходится.

\subsubsection{Задача 4: Логистическая регрессия как IRLS}

Показать, как применение метода Ньютона-Рафсона к логистической регрессии приводит к IRLS.

\textbf{Решение:}

\textit{см. раздел про логистическую регрессию}
