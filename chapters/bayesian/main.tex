
\section*{Задача 1}


\section*{Теория}

Допустим, у нас есть некоторая выборка, на которой линейные методы работают лучше решающих деревьев с точки зрения ошибки на контроле. Почему это так? Чем можно объяснить превосходство определённого метода обучения? Оказывается, ошибка любой модели складывается из трёх факторов: сложности самой выборки, схожести модели с истинной зависимостью ответов от объектов в выборке, и богатства семейства, из которого выбирается конкретная модель. Между этими факторами существует некоторый баланс, и уменьшение одного из них приводит к увеличению другого. Такое разложение ошибки носит название разложения на смещение и разброс, и его формальным выводом мы сейчас займёмся.

\vspace*{0.4cm}

Пусть задана выборка $X = (x_i, y_i)_{i=1}^n$ с вещественными ответами $y_i \in \mathbb{R}$ (рассматриваем задачу регрессии). Будем считать, что на пространстве всех объектов и ответов $X \times Y$ существует распределение $p(x, y)$, из которого сгенерирована выборка $X$ и ответы на ней.

Рассмотрим квадратичную функцию потерь
\[
L(y, a) = (y - a(x))^2
\]
и соответствующий ей среднеквадратичный риск
\[
R(a) = \mathbb{E}_{x, y} \left[ (y - a(x))^2 \right] = \int_{X} \int_{Y} p(x, y) (y - a(x))^2 dxdy.
\]

Данный функционал усредняет ошибку модели в каждой точке пространства $x$ и для каждого возможного ответа $y$, причём вклад пары $(x, y)$, по сути, пропорционален вероятности получить её в выборке $p(x, y)$. Разумеется, на практике мы не можем вычислить данный функционал, поскольку распределение $p(x, y)$ неизвестно. Тем не менее, в теории он позволяет измерить качество модели на всех возможных объектах, а не только на наблюдённой выборке.



\section*{Задание}

Покажите, что минимум среднеквадратичного риска достигается на функции, возвращающей условное математическое ожидание ответа при фиксированном объекте.

\[
a_*(x) = \mathbb{E}[y \mid x] = \int_Y y p(y \mid x) dy = \arg \min_a R(a).
\]

Иными словами покажите, что мы должны провести «взвешенное голосование» по всем возможным ответам, при этом веса ответа равны апостериорной вероятности.

\section*{Решение}

Преобразуем функцию потерь:

\[
L(y, a(x)) = (y - a(x))^2 = (y - \mathbb{E}(y \mid x) + \mathbb{E}(y \mid x) - a(x))^2 =
\]
\[
= (y - \mathbb{E}(y \mid x))^2 + 2(y - \mathbb{E}(y \mid x))(\mathbb{E}(y \mid x) - a(x)) + (\mathbb{E}(y \mid x) - a(x))^2.
\]

Подставляя её в функционал среднеквадратичного риска, получаем:

\[
R(a) = \mathbb{E}_{x,y}[L(y, a(x))] = 
\]
\[
= \mathbb{E}_{x,y}[(y - \mathbb{E}(y \mid x))^2] + \mathbb{E}_{x,y}[(\mathbb{E}(y \mid x) - a(x))^2] +
2 \mathbb{E}_{x,y}[(y - \mathbb{E}(y \mid x))(\mathbb{E}(y \mid x) - a(x))].
\]

Разберёмся сначала с последним слагаемым. Перейдём от матожидания \(\mathbb{E}_{x,y}[f(x, y)]\) к цепочке матожиданий:

\[
\mathbb{E}_x \mathbb{E}_y[f(x, y) \mid x] = \int_X \left( \int_Y f(x, y) p(y \mid x) dy \right) p(x) dx
\]

и заметим, что величина \((\mathbb{E}(y \mid x) - a(x))\) не зависит от \(y\), и поэтому её можно вынести за матожидание по \(y\):

\[
\mathbb{E}_x \mathbb{E}_y \left[ (y - \mathbb{E}(y \mid x))(\mathbb{E}(y \mid x) - a(x)) \mid x \right] =
\]
\[
= \mathbb{E}_x \left( (\mathbb{E}(y \mid x) - a(x)) \mathbb{E}_y \left[ (y - \mathbb{E}(y \mid x)) \mid x \right] \right) =
\]
\[
= \mathbb{E}_x \left( (\mathbb{E}(y \mid x) - a(x)) (\mathbb{E}_y[y \mid x] - \mathbb{E}_y \mathbb{E}(y \mid x)) \right) =
\]
\[
= 0.
\]

Получаем, что функционал среднеквадратичного риска имеет вид:

\[
R(a) = \mathbb{E}_{x,y}(y - \mathbb{E}(y \mid x))^2 + \mathbb{E}_{x,y}((\mathbb{E}(y \mid x) - a(x))^2).
\]

От алгоритма \(a(x)\) зависит только второе слагаемое, и оно достигает своего минимума, если \(a(x) = \mathbb{E}(y \mid x)\). Таким образом, оптимальная модель регрессии для квадратичной функции потерь имеет вид:

\[
a_*(x) = \mathbb{E}(y \mid x) = \int_Y y p(y \mid x) dy.
\]

Что и требовалось показать.


\section*{Задача 2}

\section*{Теория}

Для того, чтобы построить идеальную функцию регрессии, необходимо знать распределение на объектах и ответах $p(x, y)$, что, как правило, невозможно. На практике вместо этого выбирается некоторый \emph{метод обучения} $\mu : (\mathbb{X} \times \mathbb{Y})^\ell \to A$, который произвольной обучающей выборке ставит в соответствие некоторый алгоритм из семейства $A$. В качестве меры качества метода обучения можно взять усредненный по всем выборкам среднеквадратичный риск алгоритма, выбранного методом $\mu$ по выборке:
\newpage
\[
    L(\mu) = \mathbb{E}_X \left[ \mathbb{E}_{x, y} \left[ \left( y - \mu(X)(x) \right)^2 \right] \right] = \tag{1}
\]
\[  
    =\int_{(\mathbb{X} \times \mathbb{Y})^\ell} \int_{\mathbb{X} \times \mathbb{Y}} (y - \mu(X)(x))^2 
    p(x, y) \prod_{i=1}^\ell p(x_i, y_i) dx dy dx_1 dy_1 \ldots dx_\ell dy_\ell.
\]

Здесь матожидание $\mathbb{E}_X[\cdot]$ берется по всем возможным выборкам $\{(x_1, y_1), \ldots, (x_\ell, y_\ell)\}$ из распределения $\prod_{i=1}^\ell p(x_i, y_i)$.

Обратим внимание, что результатом применения метода обучения $\mu(X)$ к выборке $X$ является модель, поэтому правильно писать $\mu(X)(x)$. Но это довольно громоздкая запись, поэтому будем везде дальше писать просто $\mu(X)$, но не будем забывать, что это функция, зависящая от объекта $x$.

Среднеквадратичный риск на фиксированной выборке $X$ можно расписать как:
\[
\mathbb{E}_{x, y} \left[ \left( y - \mu(X) \right)^2 \right] = 
\mathbb{E}_{x, y} \left[ \left( y - \mathbb{E}[y \mid x] \right)^2 \right] + 
\mathbb{E}_{x, y} \left[ \left( \mathbb{E}[y \mid x] - \mu(X) \right)^2 \right].
\]

\section*{Задание}

Подставим это представление в (1):
\[
L(\mu) = \mathbb{E}_X \left[ \mathbb{E}_{x,y} \left[ \left( y - \mathbb{E}[y \mid x] \right)^2 \right] 
+ \mathbb{E}_{x,y} \left[ \left( \mathbb{E}[y \mid x] - \mu(X) \right)^2 \right] \right] =
\]
\[
= \mathbb{E}_{x,y} \left[ \left( y - \mathbb{E}[y \mid x] \right)^2 \right] 
+ \mathbb{E}_{x,y} \left[ \mathbb{E}_X \left[ \left( \mathbb{E}[y \mid x] - \mu(X) \right)^2 \right] \right]. \tag{2}
\]

Преобразуем второе слагаемое:
\[
\mathbb{E}_{x,y} \left[ 
\mathbb{E}_X \left[ \left( \mathbb{E}[y \mid x] - \mu(X) \right)^2 \right] \right] = \]
\[
= \mathbb{E}_{x,y} \left[ 
\mathbb{E}_X \left[ \left( \mathbb{E}[y \mid x] - 
\mathbb{E}_X[\mu(X)] + \mathbb{E}_X[\mu(X)] - \mu(X) \right)^2 \right] \right] =
\]
\[
= \mathbb{E}_{x,y} \left[ 
\mathbb{E}_X \left[ \left( \mathbb{E}[y \mid x] - \mathbb{E}_X[\mu(X)] \right)^2 \right] \right] 
+ \mathbb{E}_{x,y} \left[ 
\mathbb{E}_X \left[ \left( \mathbb{E}_X[\mu(X)] - \mu(X) \right)^2 \right] \right] +
\tag{3}\]
\[
+ 2 \mathbb{E}_{x,y} \left[ 
\mathbb{E}_X \left[ \left( \mathbb{E}[y \mid x] - \mathbb{E}_X[\mu(X)] \right) 
\left( \mathbb{E}_X[\mu(X)] - \mu(X) \right) \right] \right].
\]

Покажите, что последнее слагаемое обращается в нуль.

\section*{Решение}

Покажем, что последнее слагаемое обращается в нуль:
\[
\mathbb{E}_X \left[ \left( \mathbb{E}[y \mid x] - \mathbb{E}_X \left[ \mu(X) \right] \right) 
\left( \mathbb{E}_X \left[ \mu(X) \right] - \mu(X) \right) \right] =
\]
\[
= \left( \mathbb{E}[y \mid x] - \mathbb{E}_X \left[ \mu(X) \right] \right) 
\mathbb{E}_X \left[ \mathbb{E}_X \left[ \mu(X) \right] - \mu(X) \right] =
\]
\[
= \left( \mathbb{E}[y \mid x] - \mathbb{E}_X \left[ \mu(X) \right] \right) 
\left[ \mathbb{E}_X \mu(X) - \mathbb{E}_X \mu(X) \right] =
\]
\[
= 0.
\]

\section*{Задача 3}

\section*{Задание}

Используя результаты предыдущих задач и подставляя (3) в (2) получите выражение для $L(\mu)$, укажите слагаемые, отвечающие за \emph{смещение}, \emph{шум} и \emph{разброс}.

\newpage

\section*{Решение}

Подставим выражение (3) в (2), учитывая результаты предыдущих задач:

\[
L(\mu) = \underbrace{\mathbb{E}_{x, y} \left[ \left( y - \mathbb{E}[y \mid x] \right)^2 \right]}_{\text{шум}}
+ \underbrace{\mathbb{E}_x \left[ \left( \mathbb{E}_X [\mu(X)] - \mathbb{E}[y \mid x] \right)^2 \right]}_{\text{смещение}}
+ \underbrace{\mathbb{E}_x \left[ \mathbb{E}_X \left[ \left( \mu(X) - \mathbb{E}_X[\mu(X)] \right)^2 \right] \right]}_{\text{разброс}}.
\]


Рассмотрим подробнее компоненты полученного разложения ошибки. Первая компонента характеризует \emph{шум} (\emph{noise}) в данных и равна ошибке идеального алгоритма. Невозможно построить алгоритм, имеющий меньшую среднеквадратичную ошибку. Вторая компонента характеризует \emph{смещение} (\emph{bias}) метода обучения, то есть отклонение среднего ответа обученного алгоритма от ответа идеального алгоритма. Третья компонента характеризует \emph{дисперсию} (\emph{variance}), то есть разброс ответов обученных алгоритмов относительно среднего ответа.
