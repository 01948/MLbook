\section*{Байесовские методы классификации}
Решаем задачу классификации. Пусть $A = A_1 \times \ldots \times A_m$ --- пространство объектов, $B$ --- конечное множество классов. Предположим, что объекты $(x, y) \in A \times B$ независимо выбираются из какого-то неизвестного распределения с обобщённой плотностью распределения $p(x, y)$. Пусть $(X, Y)$ --- случайная величина на $A \times B$ с таким распределением. Хотим по $x$ находить его наиболее вероятный класс, то есть класс $y$, максимизирующий $P(Y=y|X=x)$. По формуле Байеса $P(y|x)=\frac{P(y)p(x|y)}{p(x)}$, поэтому максимизация $P(y|x)$ равносильна максимизации $P(y)p(x|y)$. Таким образом, задача сводится к восстанавлению дискретного априорного распределения $P(y)$ и восстановлению условного распределения $p(x|y)$.

Обычно предполагается, что $Y$ имеет произвольное категориальное распределение на $B$, то есть что о распределении $Y$ нет никакой информации, кроме множества принимаемых значений. В этом случае можно аналитически найти оценку на параметры распределения $P(Y=b_i)$ методом максимального правдоподобия.

\textbf{Задача 1.} Пусть $N$ --- количество элементов в выборке. Для любого $b \in B$ обозначим через $N_b$ --- количество элементов, для которых $y=b$ и через $\overline{p_b}$ --- частоту, с которой $y$ принимает значение $b$, то есть $\overline{p_b} = \frac{N_b}{N}$. Докажите, что $\overline{p_b}$ --- оценка максимального правдоподобия вероятностей $P(Y=b)$. \\
\textit{Указание: перейдите к максимизации логарифма правдоподобия и воспользуйтесь неравенством Гиббса.}


\subsection*{Наивный Байесовский классификатор}

Наивный Байесовский классификатор делает предположение, что признаки независимы в совокупности при условии классов, то есть для любого $k$, любых $i_1 < \ldots < i_k$ и любых $x_1, \ldots, x_k, y$ выполнено $p(X_{i_1}=x_1, \ldots, X_{i_k} = x_k|Y=y)=p(X_{i_1}=x_1|Y=y)\cdot\ldots\cdot p(X_{i_k}=x_k|Y=y)$. Отметим, что треубется именно независимость $X_i$ при условии $Y$, а не просто независимость $X_i$.

\textbf{Задача 2.} \\
a) Приведите пример совместного распределения бернуллиевских случайных величин $X_1, X_2, Y$ при котором $X_1$ и $X_2$ независимы, но не независимы при условии $Y$. \\
b) Приведите пример совместного распределения бернуллиевских случайных величин $X_1, X_2, Y$ при котором $X_1$ и $X_2$ независимы при условии $Y$, но не независимы.

Для каждого класса $b \in B$ и для каждого признака решается одномерная задача восстановления плотности $P(X_i|Y=b)$. Таким образом, предположение об условной независимости позволяет свести сложную задачу восстановления плотности многомерного распределения к более простой задаче восстановления плотности одномерного распределения.

Рассмотрим случай, когда предполагается, что условные расперделения признаков при условии класса берутся из какого-то экспоненциального семейства распределений, то есть $p(x|y) = \exp \left(\frac{\theta_y x-c(\theta_y)}{\phi_y} + h(x, \phi_y) \right)$, где $\theta_y, \phi_y$ --- параметры распределения. Параметры распределения оцениваем метода максимального правдоподобия, то есть $(\overline{\theta_y}, \overline{\phi_y}) = \text{argmax}_{\theta, \phi}\left(\sum_{i=1}^{N} \frac{\theta_yx^i-c(\theta_y)}{\phi_y} + h(x^i, \phi_y) \right)$.
Для многих распределений эта задача оптимизации решается аналитически.

В итоге после восстановления параметров получаем формулу для оценки вероятности класса
$$P(Y=y|X=x) = \frac{P(Y=y)p(X=x|Y=y)}{p(X=x)} = $$ $$ = \exp \left( \sum_{j=1}^{m} \frac{\overline{\theta_{yj}}}{\overline{\phi_{yj}}}x_j + \sum_{j=1}^{m}h(x_j, \overline{\phi_{yj}}) - \sum_{j=1}^{m}\frac{c_j(\overline{\theta_{yj}})}{\overline{\varphi_{yj}}} + \ln \overline{P(Y=y)} - \ln p(X=x) \right).$$

В случае, если $\overline{\varphi_{yj}}$ не зависит от $y$, то $\sum_{j=1}^{m}h(x_j, \overline{\phi_{yj}})$ и $\ln p(X=x)$ не зависят от $y$, поэтому максимизация вероятности класса эквивалентна максимизации $\sum_{j=1}^{m}w_{yj}x_j + w_{y0}$, где $w_{yj}=\frac{\overline{\theta_{yj}}}{\overline{\phi_{yj}}}$, $w_{y0}=\ln \overline{P(Y=y)} - \sum_{j=1}^{m}\frac{c_j(\overline{\theta_{yj}})}{\overline{\varphi_{yj}}}.$ Таким образом, в этом случае наивный байесовский классификатор строит линейную разделяющую поверхность.

\textbf{Задача 3.} Проверьте, что если все признаки бинарные, то наивный байесовский классификатор с $2$ классами эквивалентен логистической регрессии с фиксированными весами и найдите эти веса.  

\section*{Сеть радиальных базисных функций}
Сеть радиальных базисных функций - нейронная сеть прямого распространения сигнала, которая содержит промежуточный (скрытый) слой радиально симметричных нейронов. Такой нейрон преобразовывает расстояние от данного входного вектора до соответствующего ему "центра" по некоторому нелинейному закону (обычно функция Гаусса).




\subsection*{Понятие радиальной функции}

Радиальная функция — это функция f(x), зависящая только от расстояния между x и фиксированной точкой пространства X.

Для определения наших радиальных функий введем метрику:
Нормальное распределение (гауссиан) $p_j(x) = N(x; \mu _j ,\Sigma _j)$ с диагональной матрицей ковариации $\Sigma _j$ можно записать в виде


$p_j(x) = N_j exp(-1/2 \rho  _j (x, \mu _j)$



где $N_j = (2\pi)^ {-n/2}(\sigma _{j1}, \dots ,\sigma _{jn})^{-1}$ — нормировочный множитель,  

$\rho _j(x, x')$ — взвешенная евклидова метрика в n-мерном пространстве X:  

$\rho (x, x') = \sum ^n _{d = 1} \sigma ^{-2} _{jd} |\xi _d - \xi _d '|$ ,  

 $x = (\xi _1, . . . ,\xi _n), x' = (\xi _1 ', . . . , \xi _n').$

Чем меньше расстояние $\rho_j(x, \mu _j)$, тем выше значение плотности в точке x. Поэтому плотность $p _j(x)$ можно рассматривать как функцию близости вектора x к фиксированному центру $\mu_j$.

\subsection*{Постановка задачи}    

Построить алгоритм, который бы решал задачу классификации байесовским алгоритмом (частный случай EM-алгоритма) в предположении, что плотность распределения представима в виде смеси гауссовских распределений с диагональными матрицами ковариации.

\subsection*{Решение задачи}

Пусть  $|Y| = M$ - число классов, каждый класс $y \in Y$ имеет свою плотность распределения $p_y(x)$ и представлен частью выборки $X ^l _y = \{(x_i, y_i) \in X ^l | y_i = y \}.$
Здесь Y - множество ответов (классов),$y \in Y$ , $x_i$ принадлежит множеству объектов X  

\textbf{Гипотеза}

Плотности классов $p_y(x)$, $y \in Y $, представимы в виде смесей $k_y$ компонент. Каждая компонента имеет n-мерную гауссовскую плотность с параметрами 

$\mu _{yj} = (\mu _{yj1}, \dots , \mu _{yjn}) $ - центр, 

$\Sigma _{yj} = diag(\sigma  _{yj1}, \dots , \sigma  _{yjn})$ - ковариационная матрица  

$j = 1, . . . , k_y$:

 $p_y(x) = \sum ^{k _y} _{j = 1} \omega _{yj} p _{yj}(x)$,  - смесь плотностей  
 
$p_{yj}(x) = N(x; \mu _{yj} ,\Sigma _{yj})$,  - плотность каждой компоненты смеси (имеет вид гауссианы)  

 $\Sigma ^{k_y} _{j = 1} \omega _{yj} = 1, \omega _{yj} > 0$; - условия нормировки и неотрицательности весов

\textbf{Алгоритм классификации}

Запишем основную формулу байесовского классификатора $a(x) = argmax _{y \in Y} \lambda _y P _y p_y(x)$.    

Здесь Y - множество ответов (классов), x принадлежит множеству объектов X , $P_y$ - априорная вероятность класса y , $p_y(x)$ - функция правдоподобия класса y , $\lambda_{y}$ - цена ошибки на объекте класса y. Выразим плотность каждой компоненты $p_{yj}(x)$ через взвешенное евклидово расстояние от объекта x до центра компоненты $\mu _{yj}$(другими словами - подставим в основную формулу байесовского классификатора вместо $p_y(x)$ формулы, которые мы предположили в гипотезе) :


a$(x) = argmax _{y \in Y} \lambda _y P _y \sum ^{k_y} _{j = 1} N _{yj} exp(-1/2 \rho  _{yj} (x, \mu _{yj}))$


где $N _{yj} = (2\pi)^{-n/2} (\sigma _{yj1},\dots , \sigma _{yjn})^{-1}$ — нормировочные множители. Алгоритм имеет вид нейронной сети, состоящей из трёх уровней или слоёв.

Первый слой образован $k_1 + \dots+ k_M$ гауссианами $p_{yj}(x), y \in Y , j = 1, \dots, k_y$. На входе они принимают описание объекта x, на выходе выдают оценки близости объекта x к центрам $\mu _{yj}$ , равные значениям плотностей компонент в точке x.  

Второй слой состоит из M сумматоров, вычисляющих взвешенные средние этих оценок с весами $w_{yj}$ . На выходе второго слоя появляются оценки близости объекта x каждому из классов, равные значениям плотностей классов $p_{yj}(x)$.
Третий слой образуется единственным блоком argmax, принимающим окончательное решение об отнесении объекта x к одному из классов.  

Таким образом, при классификации объекта x оценивается его близость к каж- дому из центров $\mu _{yj}$ по метрике $\rho _{yj}(x, \mu _{yj}), j = 1, \dots, k_y$. Объект относится к тому классу, к чьим центрам он располагается ближе.

Описанный трёхуровневый алгоритм классификации называется сетью c радиальными базисными функциями или RBF-сетью (radial basis function network). Это одна из разновидностей нейронных сетей.

\subsection*{Обучение RBF-сети}

Обучение сводится к восстановлению плотности каждого из классов $p_y(x)$ с помощью EM-алгоритма. Результатом обучения являются центры $\mu _{yj}$ и дисперсии $\Sigma _{yj}$ компонент $j = 1, . . . , k_y$. Интересно отметить, что, оценивая дисперсии, мы фактически подбираем метрики $\rho _{yj}$ , с помощью которых будут вычисляться расстояния до центров $\mu _{yj}$ . При использовании Алгоритма, описанного в данной статье, для каждого класса определяется оптимальное число компонент смеси


\subsection*{Задачи для практики}

\textbf{Задача 1}  

Рассмотрим RBF-сеть с двумя классами $ y_1 $ и $ y_2 $. Для каждого класса задается по две компоненты смеси с центрами:
\[
\mu_{11} = (0, 0), \ \mu_{12} = (1, 1), \ \mu_{21} = (-1, 0), \ \mu_{22} = (0, -1).
\]
Ковариационные матрицы компонентов имеют одинаковую диагональную форму:
\[
\Sigma_{ij} = \begin{pmatrix} 1 & 0 \\
0 & 1 \end{pmatrix}, \ \forall i, j.
\]
Априорные вероятности классов равны $ P_{y_1} = 0.6 $ и $ P_{y_2} = 0.4 $. Весовые коэффициенты компонентов равны $ \omega_{11} = \omega_{12} = 0.5 $, $ \omega_{21} = \omega_{22} = 0.5 $. Требуется классифицировать объект $ x = (0.5, 0.5) $.

\textbf{Решение}  

1. Вычислим плотности $ p_{y_1}(x) $ и $ p_{y_2}(x) $:
\[
\rho_{11}(x, \mu_{11}) = (0.5^2 + 0.5^2) = 0.5, \ \rho_{12}(x, \mu_{12}) = (0.5 - 1)^2 + (0.5 - 1)^2 = 0.5.
\]
\[
\rho_{21}(x, \mu_{21}) = (0.5 - (-1))^2 + 0.5^2 = 2.5, \ \rho_{22}(x, \mu_{22}) = 0.5^2 + (0.5 - (-1))^2 = 2.5.
\]
2. Подставляем значения в формулу Байеса:
\[
p_{y_1}(x) = 0.5 \cdot e^{-0.5/2} + 0.5 \cdot e^{-0.5/2} = e^{-0.25},
\]
\[
p_{y_2}(x) = 0.5 \cdot e^{-2.5/2} + 0.5 \cdot e^{-2.5/2} = e^{-1.25}.
\]
3. Учитывая априорные вероятности:
\[
a(x) = \arg\max_{y \in \{y_1, y_2\}} \lambda_y P_y p_y(x).
\]
\[
P_{y_1} p_{y_1}(x) = 0.6 \cdot e^{-0.25}, \ \ P_{y_2} p_{y_2}(x) = 0.4 \cdot e^{-1.25}.
\]
Так как $ P_{y_1} p_{y_1}(x) > P_{y_2} p_{y_2}(x) $, объект относится к классу $ y_1 $.

\textbf{Задача 2}  

Дана RBF-сеть с тремя классами $ y_1, y_2, y_3 $. Пусть центры компонентов смеси для каждого класса задаются координатами:
\[
\mu_{11} = (0, 0), \ \mu_{21} = (1, 0), \ \mu_{31} = (0, 1).
\]
Все ковариационные матрицы имеют вид:
\[
\Sigma_{ij} = \begin{pmatrix} 0.5 & 0 \\
0 & 0.5 \end{pmatrix}, \ \forall i, j.
\]
Априорные вероятности классов равны $ P_{y_1} = P_{y_2} = P_{y_3} = \frac{1}{3} $. Классифицировать объект $ x = (0.7, 0.2) $.

\textbf{Решение}  

1. Рассчитаем расстояния от объекта $ x $ до каждого из центров:
\[
\rho_{11} = (0.7^2 + 0.2^2)/0.5 = 0.98, \ \rho_{21} = ((0.7 - 1)^2 + 0.2^2)/0.5 = 0.18, \ \rho_{31} = (0.7^2 + (0.2 - 1)^2)/0.5 = 1.08.
\]
2. Вычислим плотности компонентов и классов:
\[
p_{y_1}(x) = e^{-0.98/2}, \\p_{y_2}(x) = e^{-0.18/2}, \ \p_{y_3}(x) = e^{-1.08/2}.
\]
3. Учитывая равенство $ P_y $, классифицируем объект:
\[
a(x) = \arg\max_{y \in \{y_1, y_2, y_3\}} p_y(x).
\]
Наибольшая плотность у $ y_2 $, следовательно, объект относится к классу $ y_2 $.

\textbf{Задача 3}  

Пусть RBF-сеть содержит два класса $ y_1 $ и $ y_2 $ с априорными вероятностями $ P_{y_1} = 0.7 $, $ P_{y_2} = 0.3 $. Для $ y_1 $ задана одна компонента смеси с центром $ \mu_{11} = (1, 1) $ и ковариацией $ \Sigma_{11} = \begin{pmatrix} 1 & 0 \\
0 & 1 \end{pmatrix} $. Для $ y_2 $ заданы две компоненты смеси с центрами $ \mu_{21} = (0, 0) $, $ \mu_{22} = (2, 2) $ и одинаковой ковариацией $ \Sigma_{21} = \Sigma_{22} = \begin{pmatrix} 1 & 0 \\
0 & 1 \end{pmatrix} $. Найти границу между классами.

\textbf{Решение}  

1. Для $ y_1 $ плотность:
\[
p_{y_1}(x) = e^{-\rho_{11}(x, \mu_{11})/2}.
\]
2. Для $ y_2 $:
\[
p_{y_2}(x) = 0.5 e^{-\rho_{21}(x, \mu_{21})/2} + 0.5 e^{-\rho_{22}(x, \mu_{22})/2}.
\]
3. Граница определяется решением уравнения:
\[
0.7 \cdot p_{y_1}(x) = 0.3 \cdot p_{y_2}(x).
\]
Подставляя значения, решаем численно. Граница представляет собой кривую, разделяющую области максимальной плотности двух классов.
