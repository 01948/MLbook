
\section*{Разложение на смещение и разброс}


\subsection*{Теория}

Допустим, у нас есть некоторая выборка, на которой линейные методы работают лучше решающих деревьев с точки зрения ошибки на контроле. Почему это так? Чем можно объяснить превосходство определённого метода обучения? Оказывается, ошибка любой модели складывается из трёх факторов: сложности самой выборки, схожести модели с истинной зависимостью ответов от объектов в выборке, и богатства семейства, из которого выбирается конкретная модель. Между этими факторами существует некоторый баланс, и уменьшение одного из них приводит к увеличению другого. Такое разложение ошибки носит название разложения на смещение и разброс, и его формальным выводом мы сейчас займёмся.

\vspace*{0.4cm}

Пусть задана выборка $X = (x_i, y_i)_{i=1}^n$ с вещественными ответами $y_i \in \mathbb{R}$ (рассматриваем задачу регрессии). Будем считать, что на пространстве всех объектов и ответов $X \times Y$ существует распределение $p(x, y)$, из которого сгенерирована выборка $X$ и ответы на ней.

Рассмотрим квадратичную функцию потерь
\[
L(y, a) = (y - a(x))^2
\]
и соответствующий ей среднеквадратичный риск
\[
R(a) = \mathbb{E}_{x, y} \left[ (y - a(x))^2 \right] = \int_{X} \int_{Y} p(x, y) (y - a(x))^2 dxdy.
\]

Данный функционал усредняет ошибку модели в каждой точке пространства $x$ и для каждого возможного ответа $y$, причём вклад пары $(x, y)$, по сути, пропорционален вероятности получить её в выборке $p(x, y)$. Разумеется, на практике мы не можем вычислить данный функционал, поскольку распределение $p(x, y)$ неизвестно. Тем не менее, в теории он позволяет измерить качество модели на всех возможных объектах, а не только на наблюдённой выборке.

\subsection*{Задание}

Покажите, что минимум среднеквадратичного риска достигается на функции, возвращающей условное математическое ожидание ответа при фиксированном объекте.

\[
a_*(x) = \mathbb{E}[y \mid x] = \int_Y y p(y \mid x) dy = \arg \min_a R(a).
\]

Иными словами покажите, что мы должны провести «взвешенное голосование» по всем возможным ответам, при этом веса ответа равны апостериорной вероятности.

\subsection*{Решение}

Преобразуем функцию потерь:

\[
L(y, a(x)) = (y - a(x))^2 = (y - \mathbb{E}(y \mid x) + \mathbb{E}(y \mid x) - a(x))^2 =
\]
\[
= (y - \mathbb{E}(y \mid x))^2 + 2(y - \mathbb{E}(y \mid x))(\mathbb{E}(y \mid x) - a(x)) + (\mathbb{E}(y \mid x) - a(x))^2.
\]

Подставляя её в функционал среднеквадратичного риска, получаем:

\[
R(a) = \mathbb{E}_{x,y}[L(y, a(x))] = 
\]
\[
= \mathbb{E}_{x,y}[(y - \mathbb{E}(y \mid x))^2] + \mathbb{E}_{x,y}[(\mathbb{E}(y \mid x) - a(x))^2] +
2 \mathbb{E}_{x,y}[(y - \mathbb{E}(y \mid x))(\mathbb{E}(y \mid x) - a(x))].
\]

Разберёмся сначала с последним слагаемым. Перейдём от матожидания \(\mathbb{E}_{x,y}[f(x, y)]\) к цепочке матожиданий:

\[
\mathbb{E}_x \mathbb{E}_y[f(x, y) \mid x] = \int_X \left( \int_Y f(x, y) p(y \mid x) dy \right) p(x) dx
\]

и заметим, что величина \((\mathbb{E}(y \mid x) - a(x))\) не зависит от \(y\), и поэтому её можно вынести за матожидание по \(y\):

\[
\mathbb{E}_x \mathbb{E}_y \left[ (y - \mathbb{E}(y \mid x))(\mathbb{E}(y \mid x) - a(x)) \mid x \right] =
\]
\[
= \mathbb{E}_x \left( (\mathbb{E}(y \mid x) - a(x)) \mathbb{E}_y \left[ (y - \mathbb{E}(y \mid x)) \mid x \right] \right) =
\]
\[
= \mathbb{E}_x \left( (\mathbb{E}(y \mid x) - a(x)) (\mathbb{E}_y[y \mid x] - \mathbb{E}_y \mathbb{E}(y \mid x)) \right) =
\]
\[
= 0.
\]

Получаем, что функционал среднеквадратичного риска имеет вид:

\[
R(a) = \mathbb{E}_{x,y}(y - \mathbb{E}(y \mid x))^2 + \mathbb{E}_{x,y}((\mathbb{E}(y \mid x) - a(x))^2).
\]

От алгоритма \(a(x)\) зависит только второе слагаемое, и оно достигает своего минимума, если \(a(x) = \mathbb{E}(y \mid x)\). Таким образом, оптимальная модель регрессии для квадратичной функции потерь имеет вид:

\[
a_*(x) = \mathbb{E}(y \mid x) = \int_Y y p(y \mid x) dy.
\]

Что и требовалось показать.


\subsection*{Теория}

Для того, чтобы построить идеальную функцию регрессии, необходимо знать распределение на объектах и ответах $p(x, y)$, что, как правило, невозможно. На практике вместо этого выбирается некоторый \emph{метод обучения} $\mu : (\mathbb{X} \times \mathbb{Y})^\ell \to A$, который произвольной обучающей выборке ставит в соответствие некоторый алгоритм из семейства $A$. В качестве меры качества метода обучения можно взять усредненный по всем выборкам среднеквадратичный риск алгоритма, выбранного методом $\mu$ по выборке:
\newpage
\[
    L(\mu) = \mathbb{E}_X \left[ \mathbb{E}_{x, y} \left[ \left( y - \mu(X)(x) \right)^2 \right] \right] = \tag{1}
\]
\[  
    =\int_{(\mathbb{X} \times \mathbb{Y})^\ell} \int_{\mathbb{X} \times \mathbb{Y}} (y - \mu(X)(x))^2 
    p(x, y) \prod_{i=1}^\ell p(x_i, y_i) dx dy dx_1 dy_1 \ldots dx_\ell dy_\ell.
\]

Здесь матожидание $\mathbb{E}_X[\cdot]$ берется по всем возможным выборкам $\{(x_1, y_1), \ldots, (x_\ell, y_\ell)\}$ из распределения $\prod_{i=1}^\ell p(x_i, y_i)$.

Обратим внимание, что результатом применения метода обучения $\mu(X)$ к выборке $X$ является модель, поэтому правильно писать $\mu(X)(x)$. Но это довольно громоздкая запись, поэтому будем везде дальше писать просто $\mu(X)$, но не будем забывать, что это функция, зависящая от объекта $x$.

Среднеквадратичный риск на фиксированной выборке $X$ можно расписать как:
\[
\mathbb{E}_{x, y} \left[ \left( y - \mu(X) \right)^2 \right] = 
\mathbb{E}_{x, y} \left[ \left( y - \mathbb{E}[y \mid x] \right)^2 \right] + 
\mathbb{E}_{x, y} \left[ \left( \mathbb{E}[y \mid x] - \mu(X) \right)^2 \right].
\]

\subsection*{Задание}

Подставим это представление в (1):
\[
L(\mu) = \mathbb{E}_X \left[ \mathbb{E}_{x,y} \left[ \left( y - \mathbb{E}[y \mid x] \right)^2 \right] 
+ \mathbb{E}_{x,y} \left[ \left( \mathbb{E}[y \mid x] - \mu(X) \right)^2 \right] \right] =
\]
\[
= \mathbb{E}_{x,y} \left[ \left( y - \mathbb{E}[y \mid x] \right)^2 \right] 
+ \mathbb{E}_{x,y} \left[ \mathbb{E}_X \left[ \left( \mathbb{E}[y \mid x] - \mu(X) \right)^2 \right] \right]. \tag{2}
\]

Преобразуем второе слагаемое:
\[
\mathbb{E}_{x,y} \left[ 
\mathbb{E}_X \left[ \left( \mathbb{E}[y \mid x] - \mu(X) \right)^2 \right] \right] = \]
\[
= \mathbb{E}_{x,y} \left[ 
\mathbb{E}_X \left[ \left( \mathbb{E}[y \mid x] - 
\mathbb{E}_X[\mu(X)] + \mathbb{E}_X[\mu(X)] - \mu(X) \right)^2 \right] \right] =
\]
\[
= \mathbb{E}_{x,y} \left[ 
\mathbb{E}_X \left[ \left( \mathbb{E}[y \mid x] - \mathbb{E}_X[\mu(X)] \right)^2 \right] \right] 
+ \mathbb{E}_{x,y} \left[ 
\mathbb{E}_X \left[ \left( \mathbb{E}_X[\mu(X)] - \mu(X) \right)^2 \right] \right] +
\tag{3}\]
\[
+ 2 \mathbb{E}_{x,y} \left[ 
\mathbb{E}_X \left[ \left( \mathbb{E}[y \mid x] - \mathbb{E}_X[\mu(X)] \right) 
\left( \mathbb{E}_X[\mu(X)] - \mu(X) \right) \right] \right].
\]

Покажите, что последнее слагаемое обращается в нуль.

\subsection*{Решение}

Покажем, что последнее слагаемое обращается в нуль:
\[
\mathbb{E}_X \left[ \left( \mathbb{E}[y \mid x] - \mathbb{E}_X \left[ \mu(X) \right] \right) 
\left( \mathbb{E}_X \left[ \mu(X) \right] - \mu(X) \right) \right] =
\]
\[
= \left( \mathbb{E}[y \mid x] - \mathbb{E}_X \left[ \mu(X) \right] \right) 
\mathbb{E}_X \left[ \mathbb{E}_X \left[ \mu(X) \right] - \mu(X) \right] =
\]
\[
= \left( \mathbb{E}[y \mid x] - \mathbb{E}_X \left[ \mu(X) \right] \right) 
\left[ \mathbb{E}_X \mu(X) - \mathbb{E}_X \mu(X) \right] =
\]
\[
= 0.
\]

\subsection*{Задание}

Используя результаты предыдущих задач и подставляя (3) в (2) получите выражение для $L(\mu)$, укажите слагаемые, отвечающие за \emph{смещение}, \emph{шум} и \emph{разброс}.

\newpage

\subsection*{Решение}

Подставим выражение (3) в (2), учитывая результаты предыдущих задач:

\[
L(\mu) = \underbrace{\mathbb{E}_{x, y} \left[ \left( y - \mathbb{E}[y \mid x] \right)^2 \right]}_{\text{шум}}
+ \underbrace{\mathbb{E}_x \left[ \left( \mathbb{E}_X [\mu(X)] - \mathbb{E}[y \mid x] \right)^2 \right]}_{\text{смещение}}
+ \underbrace{\mathbb{E}_x \left[ \mathbb{E}_X \left[ \left( \mu(X) - \mathbb{E}_X[\mu(X)] \right)^2 \right] \right]}_{\text{разброс}}.
\]

Рассмотрим подробнее компоненты полученного разложения ошибки. Первая компонента характеризует \emph{шум} (\emph{noise}) в данных и равна ошибке идеального алгоритма. Невозможно построить алгоритм, имеющий меньшую среднеквадратичную ошибку. Вторая компонента характеризует \emph{смещение} (\emph{bias}) метода обучения, то есть отклонение среднего ответа обученного алгоритма от ответа идеального алгоритма. Третья компонента характеризует \emph{дисперсию} (\emph{variance}), то есть разброс ответов обученных алгоритмов относительно среднего ответа.

\section*{Линейный дискриминант Фишера}

Линейный дискриминант Фишера в первоначальном значении — метод, определяющий расстояние между распределениями двух разных классов объектов или событий. Он может использоваться в задачах машинного обучения при статистическом (байесовском) подходе к решению задач классификации. 

Предположим, что обучающая выборка удовлетворяет помимо базовых гипотез байесовского классификатора также следующим гипотезам:
\begin{itemize}
    \item Классы распределены по нормальному закону.
    \item Матрицы ковариаций классов равны.
\end{itemize}

Такой случай соответствует наилучшему разделению классов по дискриминанту Фишера (в первоначальном значении). Тогда статистический подход приводит к линейному дискриминанту, и именно этот алгоритм классификации в настоящее время часто понимается под термином линейный дискриминант Фишера.

\subsection*{Введение}

При некоторых общих предположениях байесовский классификатор сводится к формуле:
\[ 
a(x) = \mathrm{arg}\max_{yin Y} \lambda_{y} P_y p_y(x), 
\]
где $Y$ — множество ответов (классов), $x$ принадлежит множеству объектов $X$, $P_y$ — априорная вероятность класса $y$, $p_y(x)$ — функция правдоподобия класса $y$, $\lambda_{y}$ — весовой коэффициент (цена ошибки на объекте класса $y$).

При выдвижении всех указанных выше гипотез, кроме гипотезы о равенстве матриц ковариаций, данная формула принимает вид:
\[
a(x) = \mathrm{arg}\max_{yin Y} \left( ln(\lambda_{y} P_y) - \frac{1}{2}(x - \mu_y)^T \Sigma^{-1}_{y} (x - \mu_y) - \frac{1}{2}ln(\det{\Sigma^{-1}_{y}}) - \frac{n}{2}ln(2\pi) \right),
\]
где 
\[
\mu_y = \frac{1}{l_y} \sum^{l}_{\stackrel{i=1}{y_i = y}}x_i, \quad 
\Sigma_y = \frac{1}{l_y} \sum^{l}_{\stackrel{i=1}{y_i = y}}(x_i - \mu_y)(x_i - \mu_y)^T
\]
— приближения вектора математического ожидания и матрицы ковариации класса $y$, полученные как оценки максимума правдоподобия, $l$ — длина обучающей выборки, $l_y$ — количество объектов класса $y$ в обучающей выборке, $x \in \mathbb{R}^n$.

Данный алгоритм классификации является квадратичным дискриминантом. Он имеет ряд недостатков, одним из самых существенных из которых является плохая обусловленность или вырожденность матрицы ковариаций $\Sigma_y$ при малом количестве обучающих элементов класса $y$, вследствие чего при обращении данной матрицы $\Sigma^{-1}_{y}$ может получиться сильно искаженный результат, и весь алгоритм классификации окажется неустойчивым, будет работать плохо (возможна также ситуация, при которой обратная матрица $\Sigma^{-1}_{y}$ вообще не будет существовать). Линейный дискриминант Фишера решает данную проблему.

\textbf{Задача 1.} Каковы преимущества и недостатки использования квадратичного дискриминантного анализа (QDA) по сравнению с линейным дискриминантным анализом (LDA) в задачах классификации?


\subsection*{Основная идея алгоритма}

При принятии гипотезы о равенстве между собой ковариационных матриц алгоритм классификации принимает вид:
\[
a(x) = \mathrm{arg}\max_{yin Y} \left( ln(\lambda_{y} P_y) - \frac{1}{2}\mu_{y}^{T} \Sigma^{-1} \mu_y + x^T \Sigma^{-1} \mu_y \right),
\]
или 
\[
a(x) = \mathrm{arg}\max_{y\in Y} (\beta_y + x^T\alpha_y).
\]

Простота классификации линейным дискриминантом Фишера — одно из достоинств алгоритма: в случае с двумя классами в двумерном признаковом пространстве разделяющей поверхностью будет прямая. Если классов больше двух, то разделяющая поверхность будет кусочно-линейной. Но главным преимуществом алгоритма по сравнению с квадратичным дискриминантом является уменьшение эффекта плохой обусловленности ковариационной матрицы при недостаточных данных.

При малых $l_y$ приближения 
\[
\Sigma_y = \frac{1}{l_y} \sum^{l}_{\stackrel{i=1}{y_i = y}}(x_i - \mu_y)(x_i - \mu_y)^T
\]
дадут плохой результат, поэтому даже в тех задачах, где заведомо известно, что классы имеют различные формы, иногда бывает выгодно воспользоваться эвристикой дискриминанта Фишера и считать матрицы ковариаций всех классов одинаковыми. Это позволит вычислить некоторую "среднюю" матрицу ковариаций, используя всю выборку:
\[
\Sigma = \frac{1}{l} \sum^{l}_{i=1}(x_i - \mu_{y_i})(x_i - \mu_{y_i})^T,
\]
использование которой в большинстве случаев сделает алгоритм классификации более устойчивым.

\textbf{Задача 2.} Каковы основные предпосылки и ограничения линейного дискриминанта Фишера, и в каких случаях его применение может быть предпочтительнее по сравнению с квадратичным дискриминантом?

\subsection*{Выводы}

Эвристика линейного дискриминанта Фишера является в некотором роде упрощением квадратичного дискриминанта. Она используется с целью получить более устойчивый алгоритм классификации. Наиболее целесообразно пользоваться линейным дискриминантом Фишера, когда данных для обучения недостаточно. Вследствие основной гипотезы, на которой базируется алгоритм, наиболее удачно им решаются простые задачи классификации, в которых по формам классы "похожи" друг на друга.

Процесс классификации линейным дискриминантом Фишера можно описать следующей схемой:
\begin{enumerate}
    \item Обучение
    \begin{itemize}
        \item Оценивание математических ожиданий $\mu_y$
        \item Вычисление общей ковариационной матрицы $\Sigma$ и ее обращение
    \end{itemize}
    
    \item Классификация
    \begin{itemize}
        \item Использование формулы 
        \[
        a(x) = \mathrm{arg}\max_{yin Y} \left( ln(\lambda_{y} P_y) - \frac{1}{2}\mu_{y}^{T} \Sigma^{-1} \mu_y + x^T \Sigma^{-1} \mu_y \right)
        \]
    \end{itemize}
\end{enumerate}

\textbf{Задача 3.}Даны два класса объектов, представленные следующими данными:

\begin{itemize}
    \item Класс 1: $X_1 = {(2, 3), (3, 3), (2, 4)}$
    \item Класс 2: $X_2 = {(5, 6), (6, 5), (5, 7)}$
\end{itemize}

Найдите линейный дискриминант Фишера
\newline

\textit{Ответ к задаче 1}
\begin{itemize}
    \item QDA лучше подходит для задач, где классы имеют разные дисперсии и формы, и когда доступно достаточно данных для надежной оценки параметров.

    \item LDA может быть предпочтительнее в случаях с ограниченным количеством данных или когда классы можно считать линейно разделимыми.
\end{itemize}

\textit{Ответ к задаче 2}
Основные предпосылки линейного дискриминанта Фишера:
\begin{itemize}
    \item Нормальность: Предполагается, что данные в каждом классе распределены нормально.

    \item Однородность дисперсий: Линейный дискриминант предполагает одинаковые матрицы ковариаций для всех классов.

    \item Линейная разделимость: Предполагается, что классы можно разделить линейной границей.
\end{itemize}
Ограничения:
\begin{itemize}
    \item Если данные не удовлетворяют предпосылкам нормальности или однородности дисперсий, производительность линейного дискриминанта может значительно ухудшиться.

    \item Линейный дискриминант не может захватить сложные нелинейные зависимости между классами.
\end{itemize}
Когда предпочтительнее:
\begin{itemize}
    \item Линейный дискриминант может быть предпочтительнее квадратичного в случаях, когда:

    \item Данные имеют высокую размерность и при этом имеют достаточно малое количество образцов (линейный подход менее подвержен переобучению).

    \item Классы действительно линейно разделимы или близки к этому.
\end{itemize}
\textit{Пример:} В задачах распознавания лиц с использованием признаков (например, цветовые компоненты пикселей) линейный дискриминант может быть эффективным из-за высокой размерности данных и необходимости в простоте модели.

\textit{Ответ к задаче 3}
\begin{enumerate}
    \item Найдите средние векторы для каждого класса.
    
    Средние векторы:
    \[
    \mu_1 = \left(\frac{2+3+2}{3}, \frac{3+3+4}{3}\right) = \left(2.33, 3.33\right)
    \]
    
    \[
    \mu_2 = \left(\frac{5+6+5}{3}, \frac{6+5+7}{3}\right) = \left(5.33, 6.00\right)
    \]
    
    \item Вычислите матрицы дисперсии для каждого класса.
    
    Для класса 1:
    \[
    S_1 = \frac{1}{n_1-1} \sum_{i=1}^{n_1} (x_i - \mu_1)(x_i - \mu_1)^T
    \]
    
    После вычислений получаем:
    \[
    S_1 = \begin{pmatrix}
        0.33 & 0.33 \
        0.33 & 0.67
    \end{pmatrix}
    \]

    Для класса 2:
    \[
    S_2 = \frac{1}{n_2-1} \sum_{i=1}^{n_2} (x_i - \mu_2)(x_i - \mu_2)^T
    \]
    
    После вычислений получаем:
    \[
    S_2 = \begin{pmatrix}
        0.67 & -0.33 \
        -0.33 & 0.67
    \end{pmatrix}
    \]

    \item Найдите линейный дискриминант Фишера.
    
    Сначала находим объединённую матрицу дисперсии:
    \[
    S_W = S_1 + S_2 =
    \begin{pmatrix}
        0.33 + 0.67 & 0.33 - 0.33 \
        0.33 - 0.33 & 0.67 + 0.67
    \end{pmatrix}
    =
    \begin{pmatrix}
        1 & 0 \
        0 & 1.34
    \end{pmatrix}
    \]

    Теперь находим весовой вектор $w$:
    \[
    w = S_W^{-1}(\mu_1 - \mu_2) =
    \begin{pmatrix}
        1 & 0 \
        0 & 1.34
    \end{pmatrix}^{-1}
    \begin{pmatrix}
        2.33 - 5.33 \
        3.33 - 6
    \end{pmatrix}
    =
    \begin{pmatrix}
        -3 \
        -2.67
    \end{pmatrix}
    \]
\end{enumerate}


\section*{Байесовские методы классификации}
Решаем задачу классификации. Пусть $A = A_1 \times \ldots \times A_m$ --- пространство объектов, $B$ --- конечное множество классов. Предположим, что объекты $(x, y) \in A \times B$ независимо выбираются из какого-то неизвестного распределения с обобщённой плотностью распределения $p(x, y)$. Пусть $(X, Y)$ --- случайная величина на $A \times B$ с таким распределением. Хотим по $x$ находить его наиболее вероятный класс, то есть класс $y$, максимизирующий $P(Y=y|X=x)$. По формуле Байеса $P(y|x)=\frac{P(y)p(x|y)}{p(x)}$, поэтому максимизация $P(y|x)$ равносильна максимизации $P(y)p(x|y)$. Таким образом, задача сводится к восстанавлению дискретного априорного распределения $P(y)$ и восстановлению условного распределения $p(x|y)$.

Обычно предполагается, что $Y$ имеет произвольное категориальное распределение на $B$, то есть что о распределении $Y$ нет никакой информации, кроме множества принимаемых значений. В этом случае можно аналитически найти оценку на параметры распределения $P(Y=b_i)$ методом максимального правдоподобия.

\textbf{Задача 1.} Пусть $N$ --- количество элементов в выборке. Для любого $b \in B$ обозначим через $N_b$ --- количество элементов, для которых $y=b$ и через $\overline{p_b}$ --- частоту, с которой $y$ принимает значение $b$, то есть $\overline{p_b} = \frac{N_b}{N}$. Докажите, что $\overline{p_b}$ --- оценка максимального правдоподобия вероятностей $P(Y=b)$. \\
\textit{Указание: перейдите к максимизации логарифма правдоподобия и воспользуйтесь неравенством Гиббса.}


\subsection*{Наивный Байесовский классификатор}

Наивный Байесовский классификатор делает предположение, что признаки независимы в совокупности при условии классов, то есть для любого $k$, любых $i_1 < \ldots < i_k$ и любых $x_1, \ldots, x_k, y$ выполнено $p(X_{i_1}=x_1, \ldots, X_{i_k} = x_k|Y=y)=p(X_{i_1}=x_1|Y=y)\cdot\ldots\cdot p(X_{i_k}=x_k|Y=y)$. Отметим, что треубется именно независимость $X_i$ при условии $Y$, а не просто независимость $X_i$.

\textbf{Задача 2.} \\
a) Приведите пример совместного распределения бернуллиевских случайных величин $X_1, X_2, Y$ при котором $X_1$ и $X_2$ независимы, но не независимы при условии $Y$. \\
b) Приведите пример совместного распределения бернуллиевских случайных величин $X_1, X_2, Y$ при котором $X_1$ и $X_2$ независимы при условии $Y$, но не независимы.

Для каждого класса $b \in B$ и для каждого признака решается одномерная задача восстановления плотности $P(X_i|Y=b)$. Таким образом, предположение об условной независимости позволяет свести сложную задачу восстановления плотности многомерного распределения к более простой задаче восстановления плотности одномерного распределения.

Рассмотрим случай, когда предполагается, что условные расперделения признаков при условии класса берутся из какого-то экспоненциального семейства распределений, то есть $p(x|y) = \exp \left(\frac{\theta_y x-c(\theta_y)}{\phi_y} + h(x, \phi_y) \right)$, где $\theta_y, \phi_y$ --- параметры распределения. Параметры распределения оцениваем метода максимального правдоподобия, то есть $(\overline{\theta_y}, \overline{\phi_y}) = \text{argmax}_{\theta, \phi}\left(\sum_{i=1}^{N} \frac{\theta_yx^i-c(\theta_y)}{\phi_y} + h(x^i, \phi_y) \right)$.
Для многих распределений эта задача оптимизации решается аналитически.

В итоге после восстановления параметров получаем формулу для оценки вероятности класса
$$P(Y=y|X=x) = \frac{P(Y=y)p(X=x|Y=y)}{p(X=x)} = $$ $$ = \exp \left( \sum_{j=1}^{m} \frac{\overline{\theta_{yj}}}{\overline{\phi_{yj}}}x_j + \sum_{j=1}^{m}h(x_j, \overline{\phi_{yj}}) - \sum_{j=1}^{m}\frac{c_j(\overline{\theta_{yj}})}{\overline{\varphi_{yj}}} + \ln \overline{P(Y=y)} - \ln p(X=x) \right).$$

В случае, если $\overline{\varphi_{yj}}$ не зависит от $y$, то $\sum_{j=1}^{m}h(x_j, \overline{\phi_{yj}})$ и $\ln p(X=x)$ не зависят от $y$, поэтому максимизация вероятности класса эквивалентна максимизации $\sum_{j=1}^{m}w_{yj}x_j + w_{y0}$, где $w_{yj}=\frac{\overline{\theta_{yj}}}{\overline{\phi_{yj}}}$, $w_{y0}=\ln \overline{P(Y=y)} - \sum_{j=1}^{m}\frac{c_j(\overline{\theta_{yj}})}{\overline{\varphi_{yj}}}.$ Таким образом, в этом случае наивный байесовский классификатор строит линейную разделяющую поверхность.

\textbf{Задача 3.} Проверьте, что если все признаки бинарные, то наивный байесовский классификатор с $2$ классами эквивалентен логистической регрессии с фиксированными весами и найдите эти веса.

\section{Оценка производительности байесовских классификаторов}

\subsection{Введение}
Оценка производительности классификаторов — это важный этап в разработке моделей машинного обучения. Она позволяет понять, насколько хорошо модель справляется с задачей классификации, и выявить возможные недостатки. Для байесовских классификаторов, таких как наивный байесовский классификатор, существует несколько метрик, которые помогают оценить их эффективность.

\subsection{Основные метрики оценки производительности}

\textbf{Меткость (Accuracy)}
Определяется как доля правильно классифицированных объектов от общего числа объектов:
\[
\text{Accuracy} = \frac{TP + TN}{TP + TN + FP + FN}
\]
где:
\begin{itemize}
    \item \(TP\) — количество истинно положительных,
    \item \(TN\) — количество истинно отрицательных,
    \item \(FP\) — количество ложноположительных,
    \item \(FN\) — количество ложноотрицательных.
\end{itemize}

\textbf{Полнота (Recall)}
Показывает, какую долю положительных случаев модель смогла правильно классифицировать:
\[
\text{Recall} = \frac{TP}{TP + FN}
\]

\textbf{Точность (Precision)}
Отражает, сколько из всех объектов, классифицированных как положительные, действительно являются положительными:
\[
\text{Precision} = \frac{TP}{TP + FP}
\]

\textbf{F-мера (F1 Score)}
Гармоническое среднее между точностью и полнотой:
\[
F1 = 2 \cdot \frac{\text{Precision} \cdot \text{Recall}}{\text{Precision} + \text{Recall}}
\]

\subsection{ROC-кривая и AUC (Area Under Curve)}
ROC-кривая отображает соотношение между полнотой и ложноположительными срабатываниями при различных порогах классификации. AUC представляет собой площадь под ROC-кривой и показывает общую способность модели различать классы.

\subsection{Задачи с решениями}

\textbf{Задача 1}

\textbf{Условие}: У вас есть следующие результаты классификации:

\begin{itemize}
    \item Истинно положительные (TP): 50
    \item Истинно отрицательные (TN): 30
    \item Ложноположительные (FP): 15
    \item Ложноотрицательные (FN): 5
\end{itemize}
Рассчитайте меткость

\textbf{Решение}:
\[
\text{Accuracy} = \frac{TP + TN}{TP + TN + FP + FN} = \frac{50 + 30}{50 + 30 + 15 + 5} = \frac{80}{100} = 0.8
\]
\textbf{Ответ}: Точность классификатора составляет 0.8 (или 80\%).
\newline
\newline
\textbf{Задача 2}

\textbf{Условие}: Допустим, мы хотим оценить работу спам-фильтра почты. У нас есть 100 не-спам писем, 90 из которых наш классификатор определил верно (True Negative = 90, False Positive = 10), и 10 спам-писем, 5 из которых классификатор также определил верно (True Positive = 5, False Negative = 5). Необходимо рассчитать полноту и точность.

\textbf{Решение}:
Рассчитаем полноту:
\[
\text{Recall} = \frac{TP}{TP + FN} = \frac{5}{5 + 5} = \frac{5}{10} = 0.5
\]
Рассчитаем точность:
\[
\text{Precision} = \frac{TP}{TP + FP} = \frac{5}{5 + 10} = \frac{5}{15} \approx 0.33
\]
\textbf{Ответ}: Полнота составляет 0.5 (или 50\%), точность же составляет примерно 0.33 (или 33\%).
\newline
\newline
\textbf{Задача 3}

\textbf{Условие}: Используя результаты из предыдущей задачи, найдите F-меру.

\textbf{Решение}:
Используем значения точности и полноты, найденные ранее:
\[
F1 = 2 \cdot \frac{\text{Precision} \cdot \text{Recall}}{\text{Precision} + \text{Recall}} = 2 \cdot \frac{0.5 \cdot 0.33}{0.5 + 0.33} = 2 \cdot \frac{0.165}{0.83} \approx 0.398
\]
\textbf{Ответ}: F-мера составляет примерно 0.398 (или 39.8\%).

\section*{Сеть радиальных базисных функций}
Сеть радиальных базисных функций - нейронная сеть прямого распространения сигнала, которая содержит промежуточный (скрытый) слой радиально симметричных нейронов. Такой нейрон преобразовывает расстояние от данного входного вектора до соответствующего ему "центра" по некоторому нелинейному закону (обычно функция Гаусса).

\subsection*{Понятие радиальной функции}

Радиальная функция — это функция f(x), зависящая только от расстояния между x и фиксированной точкой пространства X.

Для определения наших радиальных функий введем метрику:
Нормальное распределение (гауссиан) $p_j(x) = N(x; \mu _j ,\Sigma _j)$ с диагональной матрицей ковариации $\Sigma _j$ можно записать в виде


$p_j(x) = N_j exp(-1/2 \rho  _j (x, \mu _j)$



где $N_j = (2\pi)^ {-n/2}(\sigma _{j1}, \dots ,\sigma _{jn})^{-1}$ — нормировочный множитель,  

$\rho _j(x, x')$ — взвешенная евклидова метрика в n-мерном пространстве X:  

$\rho (x, x') = \sum ^n _{d = 1} \sigma ^{-2} _{jd} |\xi _d - \xi _d '|$ ,  

 $x = (\xi _1, . . . ,\xi _n), x' = (\xi _1 ', . . . , \xi _n').$

Чем меньше расстояние $\rho_j(x, \mu _j)$, тем выше значение плотности в точке x. Поэтому плотность $p _j(x)$ можно рассматривать как функцию близости вектора x к фиксированному центру $\mu_j$.

\subsection*{Постановка задачи}    

Построить алгоритм, который бы решал задачу классификации байесовским алгоритмом (частный случай EM-алгоритма) в предположении, что плотность распределения представима в виде смеси гауссовских распределений с диагональными матрицами ковариации.

\subsection*{Решение задачи}

Пусть  $|Y| = M$ - число классов, каждый класс $y \in Y$ имеет свою плотность распределения $p_y(x)$ и представлен частью выборки $X ^l _y = \{(x_i, y_i) \in X ^l | y_i = y \}.$
Здесь Y - множество ответов (классов),$y \in Y$ , $x_i$ принадлежит множеству объектов X  

\textbf{Гипотеза}

Плотности классов $p_y(x)$, $y \in Y $, представимы в виде смесей $k_y$ компонент. Каждая компонента имеет n-мерную гауссовскую плотность с параметрами 

$\mu _{yj} = (\mu _{yj1}, \dots , \mu _{yjn}) $ - центр, 

$\Sigma _{yj} = diag(\sigma  _{yj1}, \dots , \sigma  _{yjn})$ - ковариационная матрица  

$j = 1, . . . , k_y$:

 $p_y(x) = \sum ^{k _y} _{j = 1} \omega _{yj} p _{yj}(x)$,  - смесь плотностей  
 
$p_{yj}(x) = N(x; \mu _{yj} ,\Sigma _{yj})$,  - плотность каждой компоненты смеси (имеет вид гауссианы)  

 $\Sigma ^{k_y} _{j = 1} \omega _{yj} = 1, \omega _{yj} > 0$; - условия нормировки и неотрицательности весов

\textbf{Алгоритм классификации}

Запишем основную формулу байесовского классификатора $a(x) = argmax _{y \in Y} \lambda _y P _y p_y(x)$.    

Здесь Y - множество ответов (классов), x принадлежит множеству объектов X , $P_y$ - априорная вероятность класса y , $p_y(x)$ - функция правдоподобия класса y , $\lambda_{y}$ - цена ошибки на объекте класса y. Выразим плотность каждой компоненты $p_{yj}(x)$ через взвешенное евклидово расстояние от объекта x до центра компоненты $\mu _{yj}$(другими словами - подставим в основную формулу байесовского классификатора вместо $p_y(x)$ формулы, которые мы предположили в гипотезе) :


a$(x) = argmax _{y \in Y} \lambda _y P _y \sum ^{k_y} _{j = 1} N _{yj} exp(-1/2 \rho  _{yj} (x, \mu _{yj}))$


где $N _{yj} = (2\pi)^{-n/2} (\sigma _{yj1},\dots , \sigma _{yjn})^{-1}$ — нормировочные множители. Алгоритм имеет вид нейронной сети, состоящей из трёх уровней или слоёв.

Первый слой образован $k_1 + \dots+ k_M$ гауссианами $p_{yj}(x), y \in Y , j = 1, \dots, k_y$. На входе они принимают описание объекта x, на выходе выдают оценки близости объекта x к центрам $\mu _{yj}$ , равные значениям плотностей компонент в точке x.  

Второй слой состоит из M сумматоров, вычисляющих взвешенные средние этих оценок с весами $w_{yj}$ . На выходе второго слоя появляются оценки близости объекта x каждому из классов, равные значениям плотностей классов $p_{yj}(x)$.
Третий слой образуется единственным блоком argmax, принимающим окончательное решение об отнесении объекта x к одному из классов.  

Таким образом, при классификации объекта x оценивается его близость к каж- дому из центров $\mu _{yj}$ по метрике $\rho _{yj}(x, \mu _{yj}), j = 1, \dots, k_y$. Объект относится к тому классу, к чьим центрам он располагается ближе.

Описанный трёхуровневый алгоритм классификации называется сетью c радиальными базисными функциями или RBF-сетью (radial basis function network). Это одна из разновидностей нейронных сетей.

\subsection*{Обучение RBF-сети}

Обучение сводится к восстановлению плотности каждого из классов $p_y(x)$ с помощью EM-алгоритма. Результатом обучения являются центры $\mu _{yj}$ и дисперсии $\Sigma _{yj}$ компонент $j = 1, . . . , k_y$. Интересно отметить, что, оценивая дисперсии, мы фактически подбираем метрики $\rho _{yj}$ , с помощью которых будут вычисляться расстояния до центров $\mu _{yj}$ . При использовании Алгоритма, описанного в данной статье, для каждого класса определяется оптимальное число компонент смеси


\subsection*{Задачи для практики}

\textbf{Задача 1}  

Рассмотрим RBF-сеть с двумя классами $ y_1 $ и $ y_2 $. Для каждого класса задается по две компоненты смеси с центрами:
\[
\mu_{11} = (0, 0), \ \mu_{12} = (1, 1), \ \mu_{21} = (-1, 0), \ \mu_{22} = (0, -1).
\]
Ковариационные матрицы компонентов имеют одинаковую диагональную форму:
\[
\Sigma_{ij} = \begin{pmatrix} 1 & 0 \\
0 & 1 \end{pmatrix}, \ \forall i, j.
\]
Априорные вероятности классов равны $ P_{y_1} = 0.6 $ и $ P_{y_2} = 0.4 $. Весовые коэффициенты компонентов равны $ \omega_{11} = \omega_{12} = 0.5 $, $ \omega_{21} = \omega_{22} = 0.5 $. Требуется классифицировать объект $ x = (0.5, 0.5) $.

\textbf{Решение}  

1. Вычислим плотности $ p_{y_1}(x) $ и $ p_{y_2}(x) $:
\[
\rho_{11}(x, \mu_{11}) = (0.5^2 + 0.5^2) = 0.5, \ \rho_{12}(x, \mu_{12}) = (0.5 - 1)^2 + (0.5 - 1)^2 = 0.5.
\]
\[
\rho_{21}(x, \mu_{21}) = (0.5 - (-1))^2 + 0.5^2 = 2.5, \ \rho_{22}(x, \mu_{22}) = 0.5^2 + (0.5 - (-1))^2 = 2.5.
\]
2. Подставляем значения в формулу Байеса:
\[
p_{y_1}(x) = 0.5 \cdot e^{-0.5/2} + 0.5 \cdot e^{-0.5/2} = e^{-0.25},
\]
\[
p_{y_2}(x) = 0.5 \cdot e^{-2.5/2} + 0.5 \cdot e^{-2.5/2} = e^{-1.25}.
\]
3. Учитывая априорные вероятности:
\[
a(x) = \arg\max_{y \in \{y_1, y_2\}} \lambda_y P_y p_y(x).
\]
\[
P_{y_1} p_{y_1}(x) = 0.6 \cdot e^{-0.25}, \ \ P_{y_2} p_{y_2}(x) = 0.4 \cdot e^{-1.25}.
\]
Так как $ P_{y_1} p_{y_1}(x) > P_{y_2} p_{y_2}(x) $, объект относится к классу $ y_1 $.

\textbf{Задача 2}  

Дана RBF-сеть с тремя классами $ y_1, y_2, y_3 $. Пусть центры компонентов смеси для каждого класса задаются координатами:
\[
\mu_{11} = (0, 0), \ \mu_{21} = (1, 0), \ \mu_{31} = (0, 1).
\]
Все ковариационные матрицы имеют вид:
\[
\Sigma_{ij} = \begin{pmatrix} 0.5 & 0 \\
0 & 0.5 \end{pmatrix}, \ \forall i, j.
\]
Априорные вероятности классов равны $ P_{y_1} = P_{y_2} = P_{y_3} = \frac{1}{3} $. Классифицировать объект $ x = (0.7, 0.2) $.

\textbf{Решение}  

1. Рассчитаем расстояния от объекта $ x $ до каждого из центров:
\[
\rho_{11} = (0.7^2 + 0.2^2)/0.5 = 0.98, \ \rho_{21} = ((0.7 - 1)^2 + 0.2^2)/0.5 = 0.18, \ \rho_{31} = (0.7^2 + (0.2 - 1)^2)/0.5 = 1.08.
\]
2. Вычислим плотности компонентов и классов:
\[
p_{y_1}(x) = e^{-0.98/2}, \\p_{y_2}(x) = e^{-0.18/2}, \ \p_{y_3}(x) = e^{-1.08/2}.
\]
3. Учитывая равенство $ P_y $, классифицируем объект:
\[
a(x) = \arg\max_{y \in \{y_1, y_2, y_3\}} p_y(x).
\]
Наибольшая плотность у $ y_2 $, следовательно, объект относится к классу $ y_2 $.

\textbf{Задача 3}  

Пусть RBF-сеть содержит два класса $ y_1 $ и $ y_2 $ с априорными вероятностями $ P_{y_1} = 0.7 $, $ P_{y_2} = 0.3 $. Для $ y_1 $ задана одна компонента смеси с центром $ \mu_{11} = (1, 1) $ и ковариацией $ \Sigma_{11} = \begin{pmatrix} 1 & 0 \\
0 & 1 \end{pmatrix} $. Для $ y_2 $ заданы две компоненты смеси с центрами $ \mu_{21} = (0, 0) $, $ \mu_{22} = (2, 2) $ и одинаковой ковариацией $ \Sigma_{21} = \Sigma_{22} = \begin{pmatrix} 1 & 0 \\
0 & 1 \end{pmatrix} $. Найти границу между классами.

\textbf{Решение}  

1. Для $ y_1 $ плотность:
\[
p_{y_1}(x) = e^{-\rho_{11}(x, \mu_{11})/2}.
\]
2. Для $ y_2 $:
\[
p_{y_2}(x) = 0.5 e^{-\rho_{21}(x, \mu_{21})/2} + 0.5 e^{-\rho_{22}(x, \mu_{22})/2}.
\]
3. Граница определяется решением уравнения:
\[
0.7 \cdot p_{y_1}(x) = 0.3 \cdot p_{y_2}(x).
\]
Подставляя значения, решаем численно. Граница представляет собой кривую, разделяющую области максимальной плотности двух классов.
